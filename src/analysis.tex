\chapter{Анализ предметной области}

\section{Основные понятия}

При моделировании важно использовать модель, адекватную исследуемой системе. Это означает, что существенные с точки зрения разработчика свойства модели и системы в достаточной для анализа степени должны совпадать. В исследованиях используют модели, а не реальные системы по следующим причинам: реальные системы очень сложны, поэтому для их анализа применяются упрощённые модели, или проведение эксперимента просто невозможно, из--за каких--либо физических ограничений. Определения основных понятий~\cite{m_types}:
\begin{itemize}[label=---]
	\item система --- совокупность объектов, взаимодействующих друг с другом, которая может является частью другой системы и включать в себя системы;
	\item модель --- объект, созданный для получения новых знаний о объекте--оригинале, отражающий только существенные свойства оригинала;
	\item моделирование --- исследование каких--либо явлений, систем или процессов путём построения и анализа модели.
\end{itemize}

На рисунке~\ref{img:m_types} представленная классификация основных видов моделирования~\cite{m_types}.
\imgScale{0.9}{m_types}{Виды моделирования}
\FloatBarrier

При \textbf{физическом} моделировании используется сама система или подобная ей. Физическая модель может быть реализована в уменьшенном или увеличенном масштабе~\cite{m_types}.

Под \textbf{математическим} моделированием понимается процесс установления соответствия реальной системе математической модели и исследование этой модели, позволяющее получить характеристики реальной системы. Применение математического моделирования позволяет исследовать объекты, реальные эксперименты над которыми затруднены или невозможны. В зависимости от вида модели математическое моделирование делится на аналитическое и компьютерное. Заметим, что аналитическое решение предпочтительнее, но его не всегда удаётся получить~\cite{m_types}.

При \textbf{компьютерном} моделировании модель формулируется в виде алгоритма или программы. Можно разделить на численное, статистическое и имитационное~\cite{m_types}.

При \textbf{численном} моделировании используются методы вычислительной математики~\cite{m_types}.

При \textbf{статистическом} моделировании выполняется обработка данных о системе с целью получения статистических характеристик системы~\cite{m_types}.

При \textbf{имитационном} моделировании процесс функционирования исследуемой системы воспроизводится на ЭВМ при соблюдении логической и временной последовательности протекания процессов, что позволяет узнать данные о состоянии системы или отдельных ее элементов в определённые моменты времени~\cite{m_types}.

\section{Моделирование многофункциональных центров обслуживания}

Многофункциональные центры обслуживания (МФЦ) --- это современные организации, призванные обеспечивать широкий спектр административных и государственных услуг гражданам и юридическим лицам в одном месте. В последние годы подобные центры получили широкое распространение в многих странах, в том числе и в России, где они известны как <<Мои документы>>. Являются структурированной системой, предназначенной для предоставления различных видов услуг клиентам. Они могут включать в себя комплексные процессы, включающие как прямое обслуживание клиентов, так и внутренние операционные процессы.

Для эффективного управления многофункциональными центрами обслуживания необходимо иметь понимание их работы и оптимальные стратегии управления. Моделирование является мощным инструментом, который позволяет анализировать и прогнозировать процессы обслуживания, а также оптимизировать их эффективность.

Услуги в центре могут оказываться как непосредственно представители организаций--участников, так и универсальными специалистами, являющимися работниками центра. Помимо этого все услуги, оказываемые на площадке МФЦ, можно разделить на три типа~\cite{serv_types}:
\begin{itemize}[label=---]
	\item консультации (результатом таких услуг является информация, за которой прошёл заявитель);
	\item приём документов (при получении таких услуг заявить приносит и отдаёт некий набор документов);
	\item выдача документов (как правило за такими услугами обращаются после первых двух, при их получении заявителю передаётся некоторый набор бумаг).
\end{itemize}

Разные типы услуг стоит рассматривать по разному в процессе моделирования.

В результате моделирования центра можно получить множество различных параметров его работы, на основе которых делать выводы об текущей эффективности и предлагать улучшения, которые также можно будет промоделировать. Такой итеративный процесс позволит создать экономичную и эффективную систему обслуживания клиентов. Выделяются следующие характеристики, имеющие практические ценность, центра, которые можно получить в результате моделирования~\cite{har1},~\cite{har2}:
\begin{itemize}[label=---]
	\item среднее время обслуживания;
	\item среднее время пребывания клиентов в очереди;
	\item вероятность простоя специалиста;
	\item вероятность попадания клиента в очередь;
	\item вероятность ухода клиента.
\end{itemize}

Исходя из этих и других характеристик можно оценить общую эффективность работы, экономическую и социальную эффективность и т.~д. Все характеристики измеряются отдельно для разных типов специалистов и очередей. 