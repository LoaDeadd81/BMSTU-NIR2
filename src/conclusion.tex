\chapter*{ЗАКЛЮЧЕНИЕ}
\addcontentsline{toc}{chapter}{ЗАКЛЮЧЕНИЕ}

Цель, которая была поставлена в начале научно--исследовательской работы, была достигнута: проведён обзор и сравнение существующих методов моделирования многофункциональных центров обслуживания.

Решены все поставленные задачи:
\begin{itemize}[label=---]
	\item изучены основные понятия моделирования многофункциональных центров обслуживания;
	\item описаны и классифицированы существующие методы;
	\item произведён сравнительный анализ рассмотренных методов.
\end{itemize}

В ходе исследования были определены особенности, преимущества и недостатки рассмотренных методов. В итоге был сделан вывод о том, что лучше всего для моделирования многофункциональных центров обслуживания подходят сети Петри, так как они позволяют моделировать стохастические и параллельные системы, а также концентрируются на локальных событиях, что позволяет получить подробную информацию о работе всех элементов системы.