\chapter[Сравнение методов моделирования многофункциональных\\центров обслуживания]{Сравнение методов моделирования\\многофункциональных центров\\обслуживания}

Исходя из данного описания моделей, используемых для моделирования многофункциональных центров обслуживания, можно сделать вывод о наиболее подходящей.

Конечные автоматы плохо подходят для данной задачи, так как работает только с полностью детерминизированными системами. Поэтому они не смогут в полной мере отобразить и проанализировать МФЦ. 

В отличие от конечных автоматов, в терминах которых описываются глобальные состояния систем, сети Петри концентрируют внимание на локальных событиях (переходах), локальных условиях (позициях) и локальных связях между событиями и условиями. Поэтому в терминах сетей Петри более адекватно, чем с помощью автоматов, модерируется поведение асинхронных систем, коими и являются многофункциональные центры обслуживания. По этой же причине и не лучшем выбором являются вероятностные автоматы, хотя они и могут моделировать стохастические процессы.

По сравнение с системами массового обслуживания сети Петри лучше представляют и отображают свойства сложных систем, также имеют возможность наглядного графического представления. Помимо этого они обладают модификацией в виде вложенных сетей Петри, которые позволяют детальнее отобразить структуру МФЦ. Также они как и СМО позволяют моделировать стохастические процессы.

Исходя из вышеприведённого сравнения можно сделать вывод о том, что сети Петри являются наилучшим выбором среди рассмотренных методов для моделирования многофункциональных центров обслуживания.