\chapter[Сравнение методов моделирования многофункциональных\\центров обслуживания]{Сравнение методов моделирования\\многофункциональных центров\\обслуживания}

Для рассмотренных методов выделим следующие критерии сравнения:
\begin{itemize}[label=---]
	\item возможность моделирования стохастических систем (К1);
	\item возможность моделирования параллельных систем (К2);
	\item тип состояния (К3) --- некоторые методы имеют только одно глобальное состояние на всю системы, другие концентрируются на локальных событиях, условиях и связях, что позволяет получить более подробную информацию о состоянии всей системы и её отдельных элементов.
\end{itemize}

Результаты сравнение приведённых методов по выделенным критериям представлены в таблице \ref{tbl:cmp}.

\begin{table}[!ht]
	\begin{center}
		\begin{threeparttable}
			\captionsetup{justification=raggedright,singlelinecheck=off}
			\caption{Результаты сравнения  методов}
			\label{tbl:cmp}
			\begin{tabular}{|l|c|c|c|c|}
				\hline
				\makecell[c]{Критерий} & КА & ВА & СМО & СП \\\hline
				К1	& Нет & Да &	Да & Да \\\hline
				К2 & Нет & Нет & Да & Да \\\hline
				К3 & Глобальное & Глобальное & Глобальное &	Локальное \\\hline
			\end{tabular}
		\end{threeparttable}
	\end{center}
\end{table}

Исходя из результатов сравнение можно сделать вывод о том, что лучшим методом для моделирования многофункциональных центров обслуживания является сеть Петри. Конечные автоматы не позволяют отобразить вероятностные события, происходящие в многофункциональных центрах обслуживания. Конечные и вероятностные автоматы не позволят отобразить параллельную работу МФЦ, т.~к. придётся вводить большое количество дополнительных состояний, что сильно усложнит модель. И в отличии от систем массового обслуживания сети Петри концентрируются на локальных событиях и имею больше одного состояния на всю систему, что позволяет получать подробную информацию о каждом элементе в системе. 